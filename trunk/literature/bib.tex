
year={1996},
title={Network optimization using evolutionary strategies},
abstract={Network optimization which has to consider both the connection distance (detour) between different nodes and the total length (costs) of the network, belongs to the class of frustrated optimization problems. Here, evolutionary strategies which include both thermodynamic and biological elements, are used to find different optimized solutions (graphs of varying density) for the network in dependence on the degree of frustration. We show, that the optimization process occurs on two different time scales, and that in the asymptotic limit a a fixed relation between the mean connection distance (detour) and the total lenght (costs) of the network exist.},
url={http://link.springer.com/chapter/10.1007/3-540-61723-X_1057?null}

year={1997},
author={Schweitzer F, Ebeling W, Rose H, Weiss O.}
title={Optimization of road networks using evolutionary strategies},
url={https://www.ncbi.nlm.nih.gov/pubmed/10021766},
abstract={A road network usually has to fulfill two requirements: (i) it should as far as possible provide direct connections between nodes to avoid large detours; and (ii) the costs for road construction and maintenance, which are assumed proportional to the total length of the roads, should be low. The optimal solution is a compromise between these contradictory demands, which in our model can be weighted by a parameter. The road optimization problem belongs to the class of frustrated optimization problems. In this paper, a special class of evolutionary strategies, such as the Boltzmann and Darwin and mixed strategies, are applied to find differently optimized solutions (graphs of varying density) for the road network, depending on the degree of frustration. We show that the optimization process occurs on two different time scales. In the asymptotic limit, a fixed relation between the mean connection distance (detour) and the total length (costs) of the network exists that defines a range of possible compromises. Furthermore, we investigate the density of states, which describes the number of solutions with a certain fitness value in the stationary regime. We find that the network problem belongs to a class of optimization problems in which more effort in optimization certainly yields better solutions. An analytical approximation for the relation between effort and improvement is derived.},