\documentclass[pdftex]{article}
% \documentclass[prl,showpacs,amsmath,amssymb]{revtex4} % PRL

% margins of 1 inch:
\setlength{\topmargin}{-.5in}
\setlength{\textheight}{9.5in}
\setlength{\oddsidemargin}{0in}
\setlength{\textwidth}{6.5in}

\usepackage[pdftex]{hyperref} % hyperlink equation and bibliographic citations
\usepackage[dvips]{graphicx,color}
\usepackage{amsmath} % advanced math
\usepackage{verbatim} % multi-line comments
\usepackage{natbib} % bibilography
\usepackage{mciteplus} % collapse multiple citations in bibilography

% from http://www.flakery.org/search/show/569

\newcommand{\infint}{\ensuremath{\int_{-\infty}^{\infty}}}

\newcommand{\ie}{\textit{i.e.}\ }
\newcommand{\eg}{\textit{e.g.}\ }

\newcommand{\eqn}[1]{Eq.\ (\ref{#1})}

\newcommand{\pfrac}[2]{\ensuremath{\frac{\partial #1}{\partial #2}}}

\begin{document}

\title{Notes}

\author{My name$^{1}$\footnote{Electronic address: username@gmail.com}
{\it $^{1}$Department of Fun, University Name \& Town, city, State Zip}}

\date{\today}

\begin{abstract}
blah blah blah
\end{abstract}

\maketitle % declares end of title page

\tableofcontents

\newpage

\section{Introduction}

Here is some text with\cite{name1,*name2} and more math $P_c=4$. All work and no play makes Jack
\begin{equation}
\begin{gathered}
y=3 \\
\int_0^x f(x)\ dx=9
\end{gathered}
\label{eq:name_of_eq}
\end{equation}

%\bibliographystyle{apsrevM} % use the apsrevM.bst for collapsible citations
%\bibliography{latex_bibliography} % external bibtex flat-file database

\end{document}
